\documentclass[root.tex]{subfiles}
\begin{document}

\section{Mise en contexte}

Les solutions présentées dans ce rapport répondent toutes au besoin d'utiliser le code directement sur un montage robotique. 
Cependant, elles présentent toutes différents avantages et inconvénients. 
Comme processus de validation, les solutions ont été testées en répondant à la même problématique: l'utilisation d'un bras robotique sans l'utilisation d'un noeud de calcul.


\section{Définition des objectifs}

Différents critères doivent être rencontrés pour qu'une solution soit acceptable et puisse substituer le système déjà en place.

\begin{enumerate}
\item Permettre la communication bidirectionnelle.
\item Permettre une communication avec un délais de moins de 10 ms. (100 Hz)
\item Permettre une grande flexibilité dans le type de commandes qu'un utilisateur peut envoyer. (commandes articulaires, cartésiennes, force, etc.)
\end{enumerate}
Les solutions présentées dans ce rapport répondent toutes à ces critères.


\newpage


\end{document}