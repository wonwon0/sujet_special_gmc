\documentclass[root.tex]{subfiles}
\begin{document}


\section{Introduction}
Dans une démarche expérimentale, il est souvent requis d'implémenter les algorithmes développés, dans un environnement tel que Matlab, sur des systèmes robotiques déjà existants. 
La solution couramment utilisée est d'employer un nœud de calcul QNX sur lequel RTLAB est installé. 
Ceci requiert un processus d'adaptation autant au niveau de l'utilisation du logiciel qu'au niveau de l'adaptation du code Matlab, python, c++ déjà développé. 
Bien souvent, le processus peut s'échelonner sur plusieurs semaines. 
Ceci motive alors la recherche d'alternatives permettant l'utilisation directe du code sur un système robotique. 
L'objectif étant de permettre l'utilisation de différents langages de programmation pour minimiser le temps d'adaptation. 
Une solution permettant l'utilisation du code Matlab, python, c++, etc. est alors mise sur pied.


\section{Note au lecteur}
Tous les documents à télécharger mentionnés dans ce rapport peuvent être trouvés dans un emplacement réseau appartenant au Laboratoire de Robotique de l'Université Laval.
Pour les obtenir de cette façon, contactez l'administrateur du serveur du laboratoire.
\end{document}